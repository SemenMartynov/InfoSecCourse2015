\newpage
\section{Проект OWASP WebGoat}

\subsection{Цель работы}

Изучить наиболее распространенные веб-уязвимости и познакомиться с OWASP WebGoat.

\subsection{Ход работы}

Для выполнения этих работ достаточно воспользоваться встроенными средствами Firefox.

\subsubsection{Недостатки контроля доступа}

В данном разделе демонстрируется группа проблем, связанных с отсутствием достаточных проверок на стороне сервера. Модифицируя поля запроса, можно выполнять операции, доступ к которым пользователя закрыт (по крайней мере нет возможности легального выполнения через интерфейс).

\subsubsection{Безопасность AJAX}

Снова проблемы фильтрации параметров, получаемых от клиента, которые встречаются в разных местах. В реальной жизни такие ошибки встречаются достаточно редко т.к. на данный момент у разработчиков достаточно автоматизированных систем тестирования для выявления подобных уязвимостей.

\subsubsection{Недостатки аутентификации}

Раздел демонстрирует очевидный факт: чем сложнее пароль, тем выше (теоретически) время его обнаружения. Фактически для взлома легче использовать уязвимости в архитектуре системы и недостаточную фильтрацию параметров, чем перебор паролей в лоб.

\subsubsection{Переполнение буфера}

Не правильная обработка сервером больших пакетов может позволить злоумышленнику получить доступ к защищаемым данным, таким как приватные поля запроса.

\subsubsection{Качество кода}

Иногда разработчики оставляют для себя рабочие комментарии, которые, попав на продакшен, способны сыграть на руку взломщику.

\subsubsection{Многопоточность}

Отладка многопоточных приложений всегда является сложной задачей. В данном разделе демонстрируется проблема с блокировками и параллельным выполнением двух запросов, которая реально имела место до некоторого времени при обработке подарочных карт в сети кофеен Starbucks.

\subsubsection{Межсайтовое выполнение сценариев}

Идея атаки  во внедрении в выдаваемую веб-системой страницу вредоносного кода. Этот код можно, к примеру, встроить в адрес... Это тоже является примером отсутствия достаточной фильтрации.

\subsubsection{Неправильная обработка ошибок}

Возникновение ошибки (к примеру, по причине отсутствия пароля) должно трактоваться системой как плохая ситуация (т.е доступ предоставляться не должен), а не наоборот.

\subsubsection{Недостатки приводящие к осуществлению инъекций (SQL и прочее)}

Недостаточная фильтрация! Параметр из адресной строки подставляется прямо в SQL-запрос, т.о. запрос можно отредактировать прямо из адресной строки.

\subsubsection{Отказ в обслуживании}

Ресурсы сервера конечны, и можно искусственно добиться их исчерпания, к примеру переполнением дискового накопителя лог-файлами.

\subsubsection{Небезопасное сетевое взаимодействие}

Передача паролей в открытом виде является плохой практикой.

\subsubsection{Небезопасная конфигурация}

Часто информацию о точке входа в интерфейс администратора можно найти в документации проекта. Там же может содержаться стандартный пользователь и пароль, который, теоретически, должен быть изменён при конфигурировании.

\subsubsection{Небезопасное хранилище}

По соли можно определить используемую хэш-функцию.

\subsubsection{Исполнение злонамеренного кода}

Если имеется возможность заливки своего файла на сервер (при этом опять же имеется не достаточная фильтрация данных), то можно легко залить свой веб-шелл.

\subsubsection{Подделка параметров}

Снова не достаточная фильтрация параметров. Пакеты можно перехватить и подменить параметры.

\subsubsection{Недостатки управление сессией}

Зная алгоритм формирования ключа в куки-файле, можно перехватить параметры и сформировать этот файл самостоятельно. Или просто его украсть.

\subsubsection{Безопасность веб-сервисов}

Снова SQL-инъекции и недостаточная фильтрация.

\subsection{Выводы}

Многие из представленных проблем хорошо известны и встречаются не часто, т.к. в данный момент имеется достаточно эффективные средства тестирования.
